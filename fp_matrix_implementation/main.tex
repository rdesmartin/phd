% This is samplepaper.tex, a sample chapter demonstrating the
% LLNCS macro package for Springer Computer Science proceedings;
% Version 2.21 of 2022/01/12
%
\documentclass[runningheads]{llncs}
%
\usepackage[T1]{fontenc}
% T1 fonts will be used to generate the final print and online PDFs,
% so please use T1 fonts in your manuscript whenever possible.
% Other font encondings may result in incorrect characters.
%
\usepackage{graphicx}
% Used for displaying a sample figure. If possible, figure files should
% be included in EPS format.
%
% If you use the hyperref package, please uncomment the following two lines
% to display URLs in blue roman font according to Springer's eBook style:
%\usepackage{color}
%\renewcommand\UrlFont{\color{blue}\rmfamily}
%
\usepackage{listings}


\usepackage{comment,cite}
\usepackage{subcaption} % For subcaptions on subfigures
\usepackage{wrapfig}

\usepackage{todonotes}
\newcommand{\knote}[1]{\todo[inline, color=blue!20]{#1}}
\newcommand{\rnote}[1]{\todo[inline, color=green!20]{#1}}


\begin{document}
%
\title{Neural Networks in Imandra: Matrix Representation as a Verification Choice\thanks{E.Komendantskaya acknowledges support of EPSRC grant EP/T026952/1.}}
%
%\titlerunning{Abbreviated paper title}
% If the paper title is too long for the running head, you can set
% an abbreviated paper title here
%
\author{Remi Desmartin\inst{1} \and
Grant Passmore\inst{2}\and
Ekaterina Kommendentskaya\inst{1}}
%
\authorrunning{R. Desmartin et al.}
% First names are abbreviated in the running head.
% If there are more than two authors, 'et al.' is used.
%
\institute{Heriot-Watt University, Edinburgh, UK\and
Imandra Inc. Austin TX, USA
\email{lncs@springer.com}\\
\url{http://www.springer.com/gp/computer-science/lncs} \and
ABC Institute, Rupert-Karls-University Heidelberg, Heidelberg, Germany\\
\email{\{abc,lncs\}@uni-heidelberg.de}}
%
\maketitle              % typeset the header of the contribution
%
\begin{abstract}
The demand for formal verification tools for neural networks has increased as neural networks have been deployed in a growing number of safety-critical applications. Matrices are a data structure essential to formalising neural networks. Functional programming languages  encourage diverse approaches to matrix definitions. This feature has already been successfully exploited in different applications. The question we ask is whether, and how, these ideas can be applied in neural network verification.
  A functional programming language Imandra combines the syntax of a functional programming language and the power of an automated theorem prover.
Using these two key features of Imandra, we explore how different implementations of matrices can influence automation of neural network verification.

\keywords{Neural networks  \and Automated reasoning \and Formal verification \and Functional programming \and Imandra.}
\end{abstract}
%
%
%

\section{Motivation}

Neural network (NN) verification was pioneered by the
SMT-solving~\cite{KaBaDiJuKo17Reluplex,HuangKWW17} and an abstract
interpretation~\cite{SinghGPV19,GeMiDrTsChVe18,AEHW20} communities. However, recently claims have been made that functional programming, too, can be valuable in this domain. 
 There is a library~\cite{MariaBLFGRG22}
formalising small rational-valued neural networks in Coq. A more sizeable
formalisation called MLCert~\cite{BS19} imports neural networks from Python, treats
floating point numbers as bit vectors, and proves properties describing the
generalisation bounds for the neural networks.
%
An $F^*$ %and \emph{Liquid Haskell}
%
formalisation~\cite{KokkeKKAA20} uses $F^*$ reals and
refinement types for proving robustness of networks trained in Python.

There are several options for defining neural networks in functional programming, ranging from
defining neurons as record types~\cite{MariaBLFGRG22} to treating them as
functions with refinement types~\cite{KokkeKKAA20}. But we claim that two general considerations
should be key to any NN formalisation choice of formalisation. Firstly, we must define neural networks
as executable functions, because we want to take advantage of executing  them   in
the functional language of choice. Secondly, a generic approach
to layer definitions is needed, particularly when we implement complex neural network architectures, such as convolutional layers.

These two essential requirements dictate that neural networks are represented as matrices, and that a programmer makes choices about matrix formalisation.
This extended abstract will explain these choices, and the consequences they imply, from the verification point of view.
We use Imandra~\cite{PassmoreCIABKKM20} to make these points, because Imandra is a functional programming language with tight integration of automated proving.

Imandra has been
successful as a user-friendly and scalable tool in the FinTech
domain~\cite{Passmore21}. The secret of its success lies in combination of the
best features of functional languages and interactive and automated theorem
provers. Imandra's logic is based on a pure, higher-order subset of OCaml, and
functions written in Imandra are at the same time valid OCaml code that can be
executed, or \emph{``simulated''}. Imandra's mode of interactive proof
development is based on a typed, higher-order lifting of the \emph{Boyer-Moore
waterfall}~\cite{BM79} for automated induction, tightly integrated with novel
techniques for SMT modulo recursive functions.


\section{Matrices in Neural Network Formalisation}
We will illusrate the functional approach to neural network formalisation and will introduce the syntax of the Imandra programming language~\cite{PassmoreCIABKKM20} by means of an example.
When we say we want to formalise neural networks as functions,
essentially, we aim to be able to define a NN using just a line of code:

\begin{lstlisting}[language=caml, label={lst:model}]
  let cnn input =
      layer_0 input >>= layer_1 >>= layer_2 >>= layer_3
\end{lstlisting}

where each \lstinline?layer_i? is defined in a modular fashion. 


 To see that a functional approach to neural networks does not necessarily imply generic nature of the code, 
let us consider an example. 
A \emph{perceptron}, also known as a \emph{linear classifier}, classifies a given input vector $X = (x_1, ..., x_m)$ into one of two classes $c_1$ or $c_2$ by computing a linear combination of the input vector with a vector of synaptic weights $(w_0, w_1, ..., w_m)$, in which $w_0$ is often called an \emph{intercept} or \emph{bias}: 
%\begin{equation}
	$ f(X) = 	\sum_{i=1}^{m}w_ix_i + w_0 $.
%\end{equation}
If the result is positive, it classifies the input as $c_1$ and if negative as $c_2$. It effectively divides the input space along a hyperplane defined by
$\sum_{i=1}^{m}w_ix_i + w_0 = 0$. 

%\begin{example}
%	Figure \ref{fig:linear_classifier} shows a 2-dimensional input space with 2 classes. In this example, the division hyperplane is the line defined by $y = ax + b$. The perceptron's only weight is a and its bias b.
%\end{example}

%\begin{figure}[ht]
%	\centering
%	\includegraphics[width=1.0\textwidth]{Figures/linear_classifier.png}
%	\rule{35em}{0.5pt}
%	\caption[Linear classifier]{The division line of a linear classifier.}
%	\label{fig:linear_classifier}
%\end{figure}

In most classification problems, classes are not linearly separated. To handle such problems, we can apply a non-linear function $a$ called an \textit{activation function} to the linear combination of weights and inputs. The resulting definition of a perceptron $f$ is:
\begin{equation}\label{eq:perceptron}
	f(X) = a\left(\sum_{i=1}^{m}w_ix_i + w_0\right)
      \end{equation}

     Let us start with a naive prototype of perceptron in Imandra.  The Iris data set is a ``Hello World'' example in data mining; it represents 3 kinds of Iris flowers using 4 selected features.
      In Imandra, inputs can be represented as a data type: % \lstinline{iris_input}

\begin{lstlisting}[language=caml]
type iris_input = {
  sepal_len: real;
  sepal_width: real;
  petal_len: real;
  petal_width: real;}
\end{lstlisting}

\begin{comment}
To process inputs of this type in the future, we can simply define

\begin{lstlisting}[language=caml]
let process_iris_input (x: iris_input) =
  let x0 = x.sepal_len in
  let x1 = x.sepal_width in
  let x2 = x.petal_len in
  let x3 = x.petal_width in
  (x1, x2, x3, x4)
\end{lstlisting}
\end{comment}

And we define a perceptron as a function:
%Finally, we assume we have an access to some trained perceptron that already has appropriate real-valued weights, and we get the following function, as expected:

\begin{lstlisting}[language=caml]
let layer_0 (w0, w1, w2, w3, w4) (x1, x2, x3, x4) =
  relu (w0 +. w1 *. x1 +. w2 *. x2 +. w3 *. x3 +. w4 *. x4)
\end{lstlisting}
where \lstinline{*.} and   \lstinline{+.} are \emph{times} and \emph{plus} defined on reals.  Note the use of the \lstinline{relu} activation function, which returns $0$ for all negative inputs and acts as the identity function otherwise.

Already in this simple example, one perceptron is not sufficient, as we must map its output to three classes. We use the usual machine learning literature trick and define a further layer of $3$ neurons, each representing one class. Each of these neurons is itself a perceptron, with one incoming weight and one bias. This gives us:

\begin{lstlisting}[language=caml]
let layer_1 (w1, b1, w2, b2, w3, b3) f1 =
  let o1 = w1 *. f1 +. b1 in
  let o2 = w2 *. f1 +. b2 in
  let o3 = w3 *. f1 +. b3 in
  (o1, o2, o3)

let process_iris_output (c0, c1, c2) =
  if (c0 >=. c1) && (c0 >=. c2) then "setosa"
  else if (c1 >=. c0) && (c1 >=. c2) then "versicolor"
  else "virginica"
\end{lstlisting}

\noindent The second function maps the output of the three neurons to the three specified classes. This post-processing stage often takes a
form of an \emph{argmax} or \emph{softmax} function, which we omit. % the definitions here for simplicity.

And thus the resulting function that defines our neural network model is:

%Suppose the given perceptron weights are\\  \lstinline{let weights_0 = (1.0023211, 1.1538234, -0.30127743, 0.9319558, 2.179688)} and\\
%\lstinline{let weights_1 = (-2.651993, 0.81521773, -0.83343804,  0.27192873, -0.27463955,  -1.21521)}, in which case this is our resulting function:

\begin{lstlisting}[language=caml]
let model input = process_iris_input input
        |> layer_0 weights_0 |> layer_1 weights_1 |>
                                        process_iris_output
\end{lstlisting}

%\textbf{Lessons Learnt.}
Although our naive formalisation has some features that we desired from the start, i.e.\ it defines a neural network as a composition of functions,
it is too inflexible to work with arbitrary compositions of layers.
%The above style of definitions would mean we will need to define all weights in all layers manually, and
In neural networks with hundreds of weights in every layer this manual approach will quickly become infeasible (as well as error prone).
So, let us generalise this attempt from the level of individual neurons to the level of matrix operations.

%\textbf{Neural Network Layers.}
%For cases when there are more classes than two, we may want to compose a perceptron with another layer, in which individual neurons represent individual classes, 
%\subsection{Multi-Layer perceptrons} \label{background_mlp}
%Fully connected feedforward NNs, also called
%and the resulting composition would be
The composition of many perceptrons is often called a \emph{multi-layer perceptron (MLP)}.
%are the best-known type of deep NNs.
An MLP consists of an input vector (also called input layer in the literature),
multiple hidden layers and an output layer, each layer 
%The hidden layers and the output layer are
made of perceptrons with weighted connections to the previous layers' outputs. %Each node is perceptron described in the previous section.
%Each layer's output is a vector of outputs of the individual perceptrons.
The weight and biases of all the neurons in a layer can be represented by two matrices denoted by $W$ and $B$. By adapting equation \ref{eq:perceptron} to this matrix notation, a layer's output $L$ can be defined as:
\begin{equation}
	L(X) = a(X \cdot W + B)
\end{equation}
where the operator $\cdot $ denotes the dot product between $X$ and each row of $W$, %i.e. the sum of the products of their corresponding elements,
$X$ is the layer's input and $a$ is the activation function shared by all nodes in a layer. %To simplify the notation, we will write $XW$ for $X*W$ in the remainder of the paper.
As the dot product multiplies pointwise all inputs by all weights, such layers are often called \emph{fully-connected}. %We will see in the next section more complex kinds of layers.

%A layer's input is the output of the previous layer.
By denoting $a_k, W_k, B_k$ --- the activation function, weights and biases of the $k$th layer respectively, an MLP $F$ with $L$ layers is traditionally defined as:
\begin{equation}
	F(X) = a_L[B_L + W_L (a_{L-1}(B_{L-1} + W_{L-1}(...(a_1(B_1+W_1\cdot X)))))]
      \end{equation}

      At this stage, we are firmly committed to using matrices and matrix operations. And we have two key choices:
\begin{enumerate}
\item to represent matrices as lists of lists (and take advantage of the inductive data type \lstinline{List}),
  \item define matrices as functions from pairs to matrix elements,
     \item  or take advantage of record types, and define matrices as records.  
\end{enumerate}

The first choice was taken in~\cite{KokkeKKAA20} (in the context of refinement types of F$^*$) and in \cite{grant_sparse_1996} (for sparse matrix encodings in Haskell), the difference between the first and second approaches was
discussed in~\cite{wood_vectors_2019} (in Agda, but with no neural network application in mind),
and the third method was taken in ~\cite{MariaBLFGRG22} using Coq (though records there were used to encode individual neurons).
In the next three sections, we will systematise these three approaches using the same formalism and the same language, in order to understand  the influence they make on neural network verification.


\section{Related Work}
\subsection{using Sized Lists or Vectors}

Grant et al. (\cite{grant_sparse_1996}) proposes 4 sparse list-based matrix implementations. They use an array-as-trees representation which allows to optimise for sparse arrays (subtrees where all the leafs are 0 are replaced by a 0-leaf).

Binary trees and lists of row-fragments: binary tree array of sparse Vectors defined as \lstinline{[(Int, [Double])]}

A generalised envelope scheme: matrix is cut up in sections
A quadtree scheme: Triangular matrix is split up in 2 triangular and a rectangular one. 
A standard quadtree structure is used for the rectangular matrix.

A Two-copy list of row-segments scheme: list of row-segments and list of column-segments in order to iterate over columns easily. Con: 2x more space is used; can be used to improve the 2 first previous methods (quadtree already bidimensional)

Pros of sparse list-based matrix representation: optimised for sparse matrices. Optimised for the specific operation considered in the paper (solving of linear systems of equations using a Cholesky scheme)

%list = inefficient random-access to data
%array = large storing size for sparse array 


\subsection{Refined Types}
%
Coq/Mathcomp/SSReflex: the size of the matrix is defined as a refinement type and used to check that matrix have the appropriate size in multiplications
Heras et al \cite{heras_incidence_2011} discuss implementation; correctness is checked at compile-time.

This is also used in Starchild/Lazuli \cite{kokke_neural_2020}.


\subsection{Arrays}
Grant et al. \cite{grant_sparse_1996} mentions using Haskell mutable arrays which are implemented using monadic operations. They stress that using a mutable array allows for modifying the array in place (thus saving memory), but it introduces "extra programming difficulties"; i.e. the use of monads makes the code less clear (as is the case in \ref{subs:monads}). 

In our case, since IML is pure, there is no access to mutable arrays.

\subsection{Monadic Operations to Check Size} \label{subs:monads}
Allows to check for valid matrix sizes when no refinement types are available. Intuitive first representation.

Drawback: introduces a lot of pattern matching, so a lot of split cases which increases the size of the program to check exponentially

Drawback: no optimisation for sparse matrices

\subsection{Matrix as Functions}
Imandra implementation; matrices are defined as total functions mapping indices to values.
Theory of uninterpreted functions.

Woods \cite{wood_vectors_2019} discusses the benefits of implementing matrices as functions in Agda.

\subsection{Matrix as Maps}

%
% ---- Bibliography ----
%
% BibTeX users should specify bibliography style 'splncs04'.
% References will then be sorted and formatted in the correct style.
%
\bibliographystyle{splncs04}
\bibliography{MatrixFormalisation}

\end{document}
