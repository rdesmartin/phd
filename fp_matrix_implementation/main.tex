% This is samplepaper.tex, a sample chapter demonstrating the
% LLNCS macro package for Springer Computer Science proceedings;
% Version 2.21 of 2022/01/12
%
\documentclass[runningheads]{llncs}
%
\usepackage[T1]{fontenc}
% T1 fonts will be used to generate the final print and online PDFs,
% so please use T1 fonts in your manuscript whenever possible.
% Other font encondings may result in incorrect characters.
%
\usepackage{graphicx}
% Used for displaying a sample figure. If possible, figure files should
% be included in EPS format.
%
% If you use the hyperref package, please uncomment the following two lines
% to display URLs in blue roman font according to Springer's eBook style:
%\usepackage{color}
%\renewcommand\UrlFont{\color{blue}\rmfamily}
%
\usepackage{listings}
\usepackage{amsfonts} 

\begin{document}
%
\title{Survey of Matrix Implementation in Functional Programming Languages\thanks{Supported by organization x.}}
%
%\titlerunning{Abbreviated paper title}
% If the paper title is too long for the running head, you can set
% an abbreviated paper title here
%
\author{Remi Desmartin\inst{1} \and
Grant Passmore\inst{2}\and
Ekaterina Kommendentskaya\inst{1}\and
Matthew Daggit\inst{1}}
%
\authorrunning{R. Desmartin et al.}
% First names are abbreviated in the running head.
% If there are more than two authors, 'et al.' is used.
%
\institute{Heriot-Watt University, Edinburgh, UK\and
Imandra Inc. Austin TX, USA
\email{lncs@springer.com}\\
\url{http://www.springer.com/gp/computer-science/lncs} \and
ABC Institute, Rupert-Karls-University Heidelberg, Heidelberg, Germany\\
\email{\{abc,lncs\}@uni-heidelberg.de}}
%
\maketitle              % typeset the header of the contribution
%
\begin{abstract}
The demand for formal verification tools for neural networks (NN) has increased as NNs have been deployed in a growing number of safety-critical applications. Matrices are a data structure essential to formalising NNs. Functional programming languages (FPLs), are particularly well-suited for automated reasoning tools which can be used for NN verification.
In this paper, we survey different implementations of matrices in functional programming languages and how they impact the verification.

\keywords{Neural networks  \and Automated reasoning \and Formal verification \and Functional programming \and Imandra.}
\end{abstract}
%
%
%
\section{Matrix as Lists of Lists}
\subsection{using Sized Lists or Vectors}

Grant et al. (\cite{grant_sparse_1996}) proposes 4 sparse list-based matrix implementations. They use an array-as-trees representation which allows to optimise for sparse arrays (subtrees where all the leafs are 0 are replaced by a 0-leaf).

Binary trees and lists of row-fragments: binary tree array of sparse Vectors defined as \lstinline{[(Int, [Double])]}

A generalised envelope scheme: matrix is cut up in sections
A quadtree scheme: Triangular matrix is split up in 2 triangular and a rectangular one. 
A standard quadtree structure is used for the rectangular matrix.

A Two-copy list of row-segments scheme: list of row-segments and list of column-segments in order to iterate over columns easily. Con: 2x more space is used; can be used to improve the 2 first previous methods (quadtree already bidimensional)

Pros of sparse list-based matrix representation: optimised for sparse matrices. Optimised for the specific operation considered in the paper (solving of linear systems of equations using a Cholesky scheme)

%list = inefficient random-access to data
%array = large storing size for sparse array 


\subsection{Refined Types}
%
Coq/Mathcomp/SSReflex: the size of the matrix is defined as a refinement type and used to check that matrix have the appropriate size in multiplications
Heras et al \cite{heras_incidence_2011} discuss implementation; correctness is checked at compile-time.

This is also used in Starchild/Lazuli \cite{kokke_neural_2020}.


\subsection{Arrays}
Grant et al. \cite{grant_sparse_1996} mentions using Haskell mutable arrays which are implemented using monadic operations. They stress that using a mutable array allows for modifying the array in place (thus saving memory), but it introduces "extra programming difficulties"; i.e. the use of monads makes the code less clear (as is the case in \ref{subs:monads}). 

In our case, since IML is pure, there is no access to mutable arrays.

\subsection{Monadic Operations to Check Size} \label{subs:monads}
Allows to check for valid matrix sizes when no refinement types are available. Intuitive first representation.

Drawback: introduces a lot of pattern matching, so a lot of split cases which increases the size of the program to check exponentially

Drawback: no optimisation for sparse matrices

\section{Matrix as Functions}
\label{section:functions}
Imandra implementation; matrices are defined as total functions mapping indices to values.
Theory of uninterpreted functions.

Woods \cite{wood_vectors_2019} discusses the benefits of implementing matrices as functions in Agda.

\section{Matrix as Maps}

Imandra's functional programming language offers the possibility to use programming language data structures. Among these, it offers access to maps and records. Combining these two data structures allow us to implement matrices in a similar way to Matrices as functions \ref{section:functions} where we map coordinates to values:
\begin{equation}
	m : \mathbb{N} \times \mathbb{N} \rightarrow \mathbb{R}
\end{equation}

In order to encode the matrix's dimensions, we use OCaml's records, which allow us to bundle several values of different types together:

\begin{lstlisting}
type 'a t = {
	rows: int;
	cols: int;
	vals: ((int*int), 'a) Map.t;
}
\end{lstlisting}

As was the case with the implementation as functions, matrices are encoded as total mapping; i.e. a matrix's value at an index where no value has been given is $0$. 

%
% ---- Bibliography ----
%
% BibTeX users should specify bibliography style 'splncs04'.
% References will then be sorted and formatted in the correct style.
%
\bibliographystyle{splncs04}
\bibliography{MatrixFormalisation}

\end{document}
